\documentclass[a4paper,12pt, titlepage]{report} % размер бумагиё устанавливаем А4, шрифт 12пунктов
\usepackage[T2A]{fontenc}
\usepackage[utf8]{inputenc} % включаем свою кодировку: koi8-r или utf8 в UNIX, cp1251 в Windows
\usepackage[english,russian]{babel} % используем русский и английский языки с переносами
\usepackage{amssymb,amsfonts,amsmath,mathtext,cite,enumerate,float} % подключаем нужные пакеты расширений
\usepackage[dvips]{graphicx} % хотим вставлять в диплом рисунки?
\usepackage{tabularx}
\usepackage{makecell}
% \usepackage{hyperref}
\graphicspath{{figure/}} % путь к рисункам

\makeatletter
\renewcommand{\@biblabel}[1]{#1.} % Заменяем библиографию с квадратных скобок на точку:
\makeatother

% пакет для расположения картинок в ряд
\usepackage{geometry} % Меняем поля страницы
\geometry{left=2.5cm} % левое поле
\geometry{right=1.5cm} % правое поле
\geometry{top=2cm} % верхнее поле
\geometry{bottom=2cm} % нижнее поле

\renewcommand{\theenumi}{\arabic{enumi}.} % Меняем везде перечисления на цифра.цифра
\renewcommand{\labelenumi}{\arabic{enumi}.} % Меняем везде перечисления на цифра.цифра
\renewcommand{\theenumii}{.\arabic{enumii}.} % Меняем везде перечисления на цифра.цифра
\renewcommand{\labelenumii}{\arabic{enumi}.\arabic{enumii}.} % Меняем везде перечисления на цифра.цифра
\renewcommand{\theenumiii}{.\arabic{enumiii}.} % Меняем везде перечисления на цифра.цифра
\renewcommand{\labelenumiii}{\arabic{enumi}.\arabic{enumii}.\arabic{enumiii}.} % Меняем везде перечисления на цифра.цифра

\usepackage[onehalfspacing]{setspace} %"умное" расстояние между строк - установить 1.5 интервала от нормального, эквивалентно

%\usepackage{floatrow} 
\usepackage{indentfirst} % Отделять первую строку раздела абзацным отступом тоже

\newcommand{\empline}{\mbox{}\newline}
\newcommand{\likechapterheading}[1]{
	\begin{center}
		\textbf{\MakeUppercase{#1}}
	\end{center}
}

\usepackage{titlesec}
\titleformat{\chapter}[display]{\filcenter}{\MakeUppercase{\chaptertitlename} \thechapter}{8pt}{\bfseries \MakeUppercase}{}\titleformat{\section}{\normalsize\bfseries}   {\thesection.}{1em}{}\titleformat{\subsection}{\normalsize\bfseries}{\thesubsection.}{1em}{}% Настройка вертикальных и горизонтальных отступов
\titlespacing*{\chapter}{0pt}{-30pt}{8pt}
\titlespacing*{\section}{\parindent}{*4}{*1}
\titlespacing*{\subsection}{\parindent}{*4}{*1}

\makeatletter
\renewcommand{\@dotsep}{2}\newcommand{\l@likechapter}[2]{{\bfseries\@dottedtocline{1}{0pt}{0pt}{#1}{#2} }} \makeatother \newcommand{\likechapter}[1]{
\newpage \clearpage \likechapterheading{#1}

\addcontentsline{toc}{likechapter}{\MakeUppercase{#1}}}

\usepackage{tocloft}
\renewcommand\contentsname{Содержание}
\renewcommand{\cfttoctitlefont}{\newpage\hspace{0.38\textwidth}\bfseries\MakeUppercase } % Оформление надписи Оглавление
\renewcommand{\cftbeforetoctitleskip}{-1em}
%\renewcommand{\cftaftertoctitle}{\mbox{}\hfill \\ \mbox{}\hfill{\footnotesize Стр.}\vspace{-2.5em}} % Добавим надпись стр.
\renewcommand{\cftchapleader}{\cftdotfill{\cftdotsep}} % Отточия после глав
\renewcommand{\cftchapaftersnum}{.} % Точка после номера главы
\renewcommand{\cftsecaftersnum}{.} % Точка после номера параграфа

\renewcommand{\cftchapfont}{\normalsize\bfseries \MakeUppercase{\chaptername} } % Оформления надписи Глава
\renewcommand{\cftsecfont}{\hspace{31pt}} % Сдвиг названия секции от левого края
\renewcommand{\cftsubsecfont}{\hspace{1pt}}
\renewcommand{\cftbeforechapskip}{1em}
\renewcommand{\cftparskip}{-1mm}  
\renewcommand{\cftdotsep}{1} \setcounter{tocdepth}{1} % Задать глубину оглавления — до subsection включительно

% Оформление подписи к рисункам и таблицам
\usepackage[font=small, tableposition=top]{caption}
\usepackage{subcaption}
\DeclareCaptionLabelFormat{gostfigure}{Рисунок #2}
\DeclareCaptionLabelFormat{gosttable}{Таблица #2}
\DeclareCaptionLabelSeparator{gost}{. }
\captionsetup{labelsep=gost}
\captionsetup[figure]{labelformat=gostfigure}
\captionsetup[table]{labelformat=gosttable}
\renewcommand{\thesubfigure}{\asbuk{subfigure}}

\usepackage[title,titletoc]{appendix}
\titleformat{\paragraph}[display]{\filcenter}{\MakeUppercase{\chaptertitlename} \thechapter}{8pt}{\bfseries}{}
\titlespacing*{\paragraph}{0pt}{-30pt}{8pt}
\newcommand{\append}[1]{\clearpage
\stepcounter{chapter}
\paragraph{\MakeUppercase{#1}}
\empline     \addcontentsline{toc}{likechapter}{\MakeUppercase{\chaptertitlename~\Asbuk{chapter}.\;#1}}}

\bibliographystyle{utf8gost705u}
%\usepackage[nottoc,notlot,notlof]{tocbibind}
\usepackage[square,numbers,sort&compress]{natbib}
\renewcommand{\bibnumfmt}[1]{#1.\hfill} % нумерация источников в самом списке — через точку
\renewcommand{\bibsection}{\likechapter{Список литературы}} % заголовок специального раздела
\setlength{\bibsep}{0pt}

\usepackage{listings}
\usepackage{color}
\usepackage{xcolor}


\lstdefinestyle{customc}{
  belowcaptionskip=1\baselineskip,
  breaklines=true,
  frame=L,
  xleftmargin=\parindent,
  showstringspaces=false,
  basicstyle=\footnotesize\ttfamily,
  keywordstyle=\bfseries\color{green!40!black},
  commentstyle=\itshape\color{purple!40!black},
  identifierstyle=\color{blue},
  stringstyle=\color{orange},
  numbers=left,               % где поставить нумерацию строк (слева\справа)
  numberstyle=\tiny  
}

\lstdefinestyle{customasm}{
  belowcaptionskip=1\baselineskip,
  frame=L,
  xleftmargin=\parindent,
  language=[x86masm]Assembler,
  basicstyle=\footnotesize\ttfamily,
  commentstyle=\itshape\color{purple!40!black},
}

\lstset{escapechar=@,style=customc}

\usepackage[utf8]{inputenc}%включаем свою кодировку: koi8-r или utf8 в UNIX, cp1251 в Windows

\usepackage{enumitem}
\makeatletter
\AddEnumerateCounter{\asbuk}{\@asbuk}{м)}
\makeatother
\setlist{nolistsep}
\renewcommand{\labelitemi}{-}
\renewcommand{\labelenumi}{\asbuk{enumi})}
\renewcommand{\labelenumii}{\arabic{enumii})}

\usepackage{amsthm}
\theoremstyle{plain}
\newtheorem{thm}{Theorem}[section]
%\newtheorem{thm}{Theorem}[subsection]
%\newtheorem{defn}[thm]{Definition}
\newtheorem{St}{Утверждение}
\newtheorem{Th}{Теорема}
\newtheorem{Cor}{Следствие}
\newtheorem{Lm}{Лемма}

\begin{document}
	\singlespacing % Одинарный междустрочный интервал
	\begin{titlepage}
	\newpage
	\begin{center}
		\includegraphics[width=0.8cm]{logo_unn_crop.eps} \\
		МИНИСТЕРСТВО  НАУКИ И ВЫСШЕГО ОБРАЗОВАНИЯ \\ 
		РОССИЙСКОЙ  ФЕДЕРАЦИИ  \\
		%\vspace{1cm}
		Федеральное государственное автономное образовательное учреждение\\* высшего образования \\*
		\textbf{<<Национальный исследовательский \\* Нижегородский государственный университет им. Н.И. Лобачевского>>}\\*
		\textbf{(ННГУ)}\\*
		\vspace{2em}
		\textbf{Институт информационных технологий, математики и механики}\\*
		\textbf{Кафедра: Теории управления и динамики систем}\\*
		\vspace{2em}
		Направление подготовки: <<Фундаментальная информатика и информационные технологии>>\\*
		Профиль подготовки: <<Когнитивные системы>>\\*
	\end{center}

	\vspace{1em}

	\begin{center}
		{\Large \textbf{МАГИСТЕРСКАЯ ДИССЕРТАЦИЯ}}
    \end{center}

    \vspace{2.5em}

	\begin{center}
		{\textbf{На тему:}}\\
		
		{\large \textbf{<<Кластерные вращательные режимы в системе глобально связанных фазовых осцилляторов с инерцией>>}}
	\end{center}

    \vspace{4em}

    \hfill
    \begin{minipage}[t]{.36\linewidth}
    	\begin{flushleft}
    	
			\textbf{Выполнил:} студент группы 382006-3м \\
			% \vspace{1.0em}
			Хорькин Дмитрий Сергеевич\\
			\vspace{1.0em}
			\hrulefill  \\
			{\small подпись}\\
			
			\vspace{1.5em}
			\textbf{Научный руководитель:}\\
		зав. лаб., к.ф.-м.н.\\
			% \vspace{1.0em}
			Смирнов Лев Александрович\\
			\vspace{1.0em}
			\hrulefill  \\
			{\small подпись}
			
    	\end{flushleft}
    \end{minipage}

	\vspace{\fill}

	\begin{center}
		Нижний Новгород \\
		2022
	\end{center}

\end{titlepage} % Это титульный лист
	\setcounter{page}{2} % Нумерация страниц со 2-й, 1-й является титульная страница
	\onehalfspacing % Полуторный междустрочный интервал
	\setlength{\parindent}{1.25cm} % Абзацный отступ

	\likechapter{Аннотация}
В данной работе рассматриваются двухкластерные вращательные режимы в системе Курамото с инерцией. 
В частности, изучается вопрос существования и устойчивости двухкластерных вращательных режимов в зависимости от управляющий параметров.
Также был реализован программный комплекс, позволяющий эффективно находить интересующие вращательные
режимы в произвольных системах связанных элементов.
Было разработано web-приложение, выполняющее прямое численное моделирование системы.
Аналитические результаты подтверждены прямым численным моделированием.

	\def\contentsname{Содержание} % "Содержание" вместо "Оглавление"
	\def\bibname{Список литературы}
	\tableofcontents % Это оглавление, которое генерируется автоматически
	\likechapter{Введение}
Сети фазовых осцилляторов часто используются для моделирования
совместной динамики в различных биологических и
искусственных системах, начиная от нейронных сетей \cite{Hoppensteadt:Izhikevich} и
популяций химических осцилляторов \cite{Tinsley:Nkomo} до лазерных решеток \cite{Ding:Belykh}
и электрических сетей \cite{Dorfler:Chertkov}. Система
фазовых осцилляторов Курамото первого порядка \cite{Kuramoto,Strogatz} представляет собой широко адаптированную модель
сети фазовых осцилляторов, которая может демонстрировать сложную
пространственно-временную динамику при переходе от
некогерентности к полной синхронизации \cite{Acebron:Bonilla,Barreto:Hunt,Ott:Antonsen,Hong:Chate,Pikovsky:Rosenblum,Maistrenko:Popovych,Dorfler:Bullo,Martens:Barreto}. Когда
колебания в модели Курамото имеют неоднородные частоты, этот переход обычно
сопровождается частичной синхронизацией, которая возникает, когда система
распадается на кластеры когерентных и некогерентных осцилляторов
[7, 14, 15]. В случае идентичных осцилляторов частичная
синхронизация может превратиться в химерные состояния,
которые представляют собой захватывающие паттерны, в которых даже
структурно идентичные осцилляторы могут распадаться на две, возможно,
асимметричные группы когерентных и некогерентных осцилляторов [16-19].
Химерные состояния были тщательно изучены в модели Курамото,
а также в других сетях колебательных систем [19-28], включая связанные химические генераторы [2], сети
метрономов [29], связанные маятники [30],
педали на мосту [31], оптические системы и лазеры [32]
и непрерывные среды [33, 34].

Целью данной работы является изучение двухкластерных вращательных режимов в системе Курамото
с инерцией и фазовой задержкой, для успешного достижения которой выделены следующие задачи: \\
1) Исследование существования и типов возникающих двухкластерных вращательных режимов в зависимости от управляющих параметров. \\
2) Исследование устойчивости двухкластерных вращательных режимов. \\
3) Реализация программного комплекса, позволяющего эффективно находить интересующие вращательные движения, а также определять
их устойчивость в произвольных системах связанных элементов.

	\begin{chapter}{Описание модели. Кластерные режимы}
	\section{Модель Курамото с инерцией}
	В данной работе исследуется модель глобально связанных осцилляторов Курамото второго порядка, описываемая системой уравнений:
	\begin{equation} \label{main-sistem}
		m\ddot{\varphi}_i + \dot{\varphi}_i = \omega + 
		\frac{1}{N} \sum_{j = 1}^N \sin{(\varphi_j - 
		\varphi_i - \alpha)}, i = \overline{1, N}, 
	\end{equation}
	где $m$ - параметр инерции, $\omega$ - постоянный вращающий момент,
	$\alpha$ - фазовая задержка, $N$ - общее число элементов.
	Предполагается, что осцилляторы являются идентичными, с частотой $\omega$, массой $m$ и запаздыванием по
	фазе $\alpha \in [0, \pi)$ \cite{Sakaguchi} . В системе \eqref{main-sistem} имеется синфазный режим,
	который локально устойчив для любого $\alpha \in [0, \pi/2)$
	и неустойчив для любого $\alpha \in [\pi/2, \pi)$ \cite{Acebron:Bonilla}. Поэтому принято считать связь притягивающей при
	$\alpha < \pi/2$ и отталкивающей при $\pi/2 \leq \alpha < \pi$.

	\section{Двухкластерные вращательные режимы}
	
	Двухкластерный режим в модели \eqref{main-sistem} описывается системой:
	
	\begin{equation} \label{two-cluster}
		\begin{cases}
			m\ddot{\psi}_1 + \dot{\psi}_1 = \omega + \frac{N-K}{N} \sin{(\psi_2 - \psi_1 - \alpha)} - \frac{K}{N}\sin{\alpha},\\
			m\ddot{\psi}_2 + \dot{\psi}_2 = \omega + \frac{K}{N} \sin{(\psi_1 - \psi_2 - \alpha)} - \frac{N - K}{N}\sin{\alpha},
		\end{cases}
	\end{equation}
	где $\psi_1$, $\psi_2$ - фазы элементов первого и
	второго кластера, соответственно, $K$ - количество элементов в малом кластере.
	Введем новую переменную $\beta = \frac{K}{N}$, характеризующую долю элементов малого кластера относительно размера $N$ всего ансамбля.
	Уравнение \eqref{two-cluster} перепишется в виде:
	
	\begin{equation} \label{two-cluster-beta}
		\begin{cases}
			m\ddot{\psi}_1 + \dot{\psi}_1 = \omega + (1 - \beta) \sin{(\psi_2 - \psi_1 - \alpha)} - \beta\sin{\alpha}, \\
			m\ddot{\psi}_2 + \dot{\psi}_2 = \omega + \beta \sin{(\psi_1 - \psi_2 - \alpha)} - (1 - \beta)\sin{\alpha}.
		\end{cases}
	\end{equation}
	
	Введем величину $X = \psi_1 - \psi_2$, характеризующую расстройку фаз между элементами двух кластеров.
	Вычитая в \eqref{two-cluster-beta} из первого уравнения второе, приходим к уравнению для отстройки $X$:

	\begin{equation} \label{pre-pend}
		m\ddot{X} + \dot{X} = (1 - 2 \beta) \sin{\alpha} - \left[(1-\beta)\sin{(X + \alpha)} + \beta\sin{(X - \alpha)} \right].
	\end{equation}
	
	Заметим, что решение $X = 0$ соответствует реализации
	синфазного вращательного режима. Ненулевое решение $X \neq 0$ соответствует
	реализации двухкластерного режима.
	
	
	Для дальнейшего анализа уравнения \eqref{pre-pend} произведем ряд замен, а именно:
	\begin{align*}
	R^2 = (N - 2K)^2 \sin{\alpha}^2 + N^2 \cos{\alpha}^2, \\
	\rho = \sqrt{\frac{N}{m R}}, \ t = \hat{t} \frac{N}{\rho R}, \gamma = \frac{N - 2K}{R}\sin{\alpha}, \\
	\Phi = X + \delta, \ \delta = \arccos{(\frac{N}{R}\cos{\alpha})}, \\
	\end{align*}
	что приводит к:
	
	\begin{equation} \label{pend}
		\frac{d^2 \Phi }{d\hat{t}^2} + \rho \frac{d\Phi}{d\hat{t}} + \sin{\Phi} = \gamma.
	\end{equation}

	\begin{figure}[h!]
		\begin{center}
			\includegraphics[width=0.7\columnwidth]{pictures/bf-tricommy.png}
		\end{center}
		\caption{Бифуркационная диаграмма маятникового уравнения \eqref{pend}}
		\label{bf-d}
	\end{figure}

	Полученное уравнение \eqref{pend} хорошо изучено в работе \cite{Andronov:Vitt}.
	В зависимости от параметров $\rho$ и $\gamma$, в системе \eqref{pend}
	могут существовать два состояния равновесия:
	устойчивая точка $\Phi_e = \arcsin{\gamma}$ и седло
	$\Phi_s = \pi - \arcsin{\gamma}$, а также
	некоторое устойчивое периодическое вращательное движение.
	Кривая $\gamma=T(\rho)$ называется кривой Трикоми и определяет границу области существования вращательного движения (области B и C).
	% Данные соотношения определяются так называемой
	% бифуркационной кривой Трикомми (см. рис. \ref{bf-d}).
	Таким образом, в исходной системе могут существовать 2 типа
	двухкластерных вращательных режимов, первый характеризуется
	постоянной расстройкой фаз, второй характеризуется
    периодической на цилиндре расстройкой фаз.
	Область существования вращательного движения в уравнении \eqref{pend}
	определяется соотношением:
	\begin{equation}
		\gamma \ge T(\rho),
		\label{eroi}
	\end{equation}
	где $T(.)$ - уравнение кривой Трикоми. В качестве аппроксимации для кривой Трикоми можно использовать выражение
	$T(\rho) = \rho\frac{4}{\pi} - 0.305\rho^3$, полученное в \cite{Belykh:Brister}.
	Подставляя данную аппроксимацию кривой Трикому в уравнение \eqref{eroi} получаем уравнения для границы области существования двухкластерного вращательного движения с
	периодической расстройкой фаз в области параметров $\alpha$, $m$:
	\begin{equation} \label{borders}
		m = \frac{1}{\sqrt{(1 - 2\beta)^2\sin{\alpha}^2 + \cos{\alpha}^2} \cdot T^{-1}(\frac{(1 - 2\beta)\sin{\alpha}}{\sqrt{(1 - 2\beta)^2\sin{\alpha}^2 + \cos{\alpha}^2}})}.
	\end{equation}

	Заметим, что двухкластерный вращательный режим при $\beta = 0.5$ не существует.

	\begin{figure}[h!]
		\begin{center}
			\includegraphics[width=1\columnwidth]{pictures/ex.png}
		\end{center}
		\caption{Границы зон устойчивости двухкластерного вращательного режима с вращающейся расстройкой фаз в зависимости от параметра $\beta$}
		\label{ex-zones}
	\end{figure}

	На рисунке \ref{ex-zones} изображены границы зон устойчивости двухкластерного вращательного режима
	с вращающейся расстройкой фаз в зависимости от параметра $\beta$.
	Можно заметить, что при фиксированном $N$ зона существования двухкластерного режима при размере малого кластера равного $K_1$ полностью включена в 
	зону существования двухкластерного режима при размере малого кластера равного $K_2$ при $K_1 > K_2$. Наблюдается мультистабильность вращательных режимов.


	На рисунке \ref{schema} изображены двухкластерные вращательные режимы в зависимости от параметра $N$.
	Каждая ячейка представляет собой определенный двухкластерный режим,
	характеризующийся параметром $\beta$, который записан в виде дроби в центре ячейки. Ячейки с одинаковым цветом, кроме серого, представляют собой
	один и тот же двухкластерный вращательный режим. Несложно заметить, что при фиксированном $N = N^*$
	двухкластерный вращательный режим, соответствующий $\beta^* = A/B$, где $A/B$ - несократимая дробь, 
	будет повторяться для всех $N = N^* + k\cdot B$. 

	\begin{figure}[h!]
		\begin{center}
			\includegraphics[width=1\columnwidth]{pictures/schema.png}
		\end{center}
		\caption{Двухкластерные вращательные режимы в зависимости от параметра $N$. Каждая ячейка представляет собой определенный двухкластерный режим,
		характеризующийся параметром $\beta$, который записан в виде дроби в центре ячейки. Ячейки с одинаковым цветом, кроме серого, представляют собой
		один и тот же двухкластерный вращательный режим}
		\label{schema}
	\end{figure}

\end{chapter}
	\begin{chapter}{Устойчивость двухкластерных вращательных режимов}
	\section{Устойчивость двухкластерных вращательных движений с постоянной расстройкой фаз}
	
	Перепишем \ref{pre-pend} в виде системы:
	
	\begin{equation} \label{system}
		\begin{cases}
			\dot{x} = y, \\
			\dot{y} = \frac{1}{m} \left[ (1 - 2\beta)\sin{\alpha}(1 - \cos{x}) - \sin{x}\cos{\alpha} - y \right]
		\end{cases}
	\end{equation}
	
	Выполним линеаризацию системы \ref{system}:
	
	\begin{equation} \label{system-linear}
		\begin{pmatrix}
			\dot{\hat{x}} \\
			\dot{\hat{y}}
		\end{pmatrix}
		=
		\begin{pmatrix}
			0 & 1 \\
			\frac{1}{m}\left[ (1 - 2\beta)\sin{\alpha}\sin{x_p} - \cos{\alpha}\cos{x_p} \right] & -\frac{1}{m}
		\end{pmatrix}
		\begin{pmatrix}
			\hat{x} \\
			\hat{y}
		\end{pmatrix}
	\end{equation}
	Где $x_p$ - стационарные состояния системы \ref{pre-pend}. Характеристический многочлен системы \ref{system-linear}:
	
	\begin{equation} \label{hp}
		\lambda^2 + \frac{1}{m}\lambda - \frac{1}{m}\left[ (1 - 2\beta)\sin{\alpha}\sin{x_p} - \cos{\alpha}\cos{x_p} \right] = 0
	\end{equation}
	
	Из \ref{hp} можно заметить, что устойчивость стационарного состояния $x_p$ определяется соотношением:
	
	\begin{equation} \label{hp-stability}
		\cos{\alpha}\cos{x_p} - (1 - 2\beta)\sin{\alpha}\sin{x_p} > 0
	\end{equation}
	
	Несложно заметить, что стационарные состояния в уравнении \ref{pre-pend} определяются выражением:
	$$
	\sin{(x_p + \varphi)} = \frac{(1 - 2\beta) \sin{\alpha}}{\sqrt{\cos{\alpha}^2 + (1 - 2\beta)^2\sin{\alpha}^2}},
	$$
	Откуда:
	\begin{equation} \label{x1}
		x_{p_1} = \begin{cases}
			0, \alpha \in [-\pi/2, \pi/2) \\
			2\arcsin{\frac{(1 - 2\beta) \sin{\alpha}}{\sqrt{\cos{\alpha}^2 + (1 - 2\beta)^2\sin{\alpha}^2}}} - \pi , \alpha \in [\pi/2, 3\pi/2)
		\end{cases}
	\end{equation}
	
	\begin{equation} \label{x2}
	x_{p_2} = \begin{cases}
		\pi - 2\arcsin{\frac{(1 - 2\beta) \sin{\alpha}}{\sqrt{\cos{\alpha}^2 + (1 - 2\beta)^2\sin{\alpha}^2}}}, \alpha \in [-\pi/2, \pi/2) \\
		0, \alpha \in [\pi/2, 3\pi/2)
		\end{cases}
	\end{equation}
	
	Подставляя \ref{x1}, \ref{x2} в \ref{hp-stability}
	получаем, что $x_{p_1}$ - устойчива, $x_{p_2}$ - неустойчива.

	\begin{figure}[h!]\center
		\begin{tabular}{cc}
		\includegraphics[width=0.54\columnwidth]{pictures/fixed-points.png}
		&
		\includegraphics[width=0.5\columnwidth]{pictures/fixed-points-2.png} 
		\end{tabular}
		\caption{\textbf{Устойчивость стационарных состояний.}
		Синяя пунктирная линия - устойчивое состояние $x_{p_1}$,
		Красная сплошная линия - неустойчивое состояние $x_{p_2}$.
		$\beta = 0.3$, $\beta = 0.4$}
	\end{figure}

	\begin{figure}[h!]
		\begin{center}
			\includegraphics[width=1\columnwidth]{pictures/fixed-points-3.png}
		\end{center}
		\caption{\textbf{Устойчивость стационарных состояний.}
		Синяя пунктирная линия - устойчивое состояние $x_{p_1}$,
		Красная сплошная линия - неустойчивое состояние $x_{p_2}$.
		$\beta = 0.494$}
	\end{figure}

	Заметим, что при $\beta$ стремящейся к 0.5 стационарные состояния с расстройкой фаз неравной нулю стремятся
	к $\pi k$, $k \in \mathbb{Z}$.
	Стоит отметить, что в полной системе \ref{main-sistem} стационарное двухкластерное состояние $x_{p_1}$
	может терять свою устойчивость. Данный эффект будет продемонстрирован ниже.
	
	\section{Устойчивость двухкластерных вращательных движений с периодической расстройкой фаз}

	Выполним линеаризацию относительно произвольного вращательного движения $\psi_i$ с помощью замены, где вариации $\delta_i$ малы:
	$\varphi_i = \psi_i + \delta_i$:
	\begin{equation} \label{pretubr}
		m\ddot{\delta}_i + \dot{\delta}_i = \frac{1}{N} \sum_{j = 1}^N \cos{(\psi_j - \psi_i - \alpha)} \cdot (\delta_j - \delta_i), \ i = \overline{1, N}
	\end{equation}
	Для случая двухкластерного режима \ref{two-cluster} система \ref{pretubr} запишется в виде:
	
	\begin{equation}
		\begin{cases}
			m\ddot{\delta}_i + \dot{\delta}_i = \frac{1}{N} \left( \cos{\alpha} \sum_{j = 1}^K (\delta_j - \delta_i) + \cos{(X + \alpha)} \sum_{j = K + 1}^N (\delta_j - \delta_i) \right), \ i = \overline{1,K}, \\
			m\ddot{\delta}_i + \dot{\delta}_i = \frac{1}{N} \left( \cos{(X - \alpha)} \sum_{j = 1}^K (\delta_j - \delta_i) +  \cos{\alpha} \sum_{j = K + 1}^N (\delta_j - \delta_i)  \right), \ i = \overline{K + 1,N},
		\end{cases}		
	\end{equation}
	
	Выполняя замену:
	\begin{align*}
		\eta_1 = \frac{1}{K} \sum_{i = 1}^K \delta_i - \frac{1}{N - K} \sum_{i = K + 1}^N \delta_i, \\
		\eta_2 = \frac{1}{K} \sum_{i = 1}^K \delta_i + \frac{1}{N - K} \sum_{i = K + 1}^N \delta_i, \\
		\xi_n = \delta_{n+1} - \delta_n, \ 1 \leq  n \leq K - 1, \\
		\zeta_n = \delta_{n+1} - \delta_n, \ K + 1 \leq n \leq N - 1.
	\end{align*}
		
	Приходим к системе уравнений:
	
	\begin{equation} \label{split-linear-pert-sys-n12}
		\begin{cases}
			m\ddot{\eta}_1 + \dot{\eta}_1 + \left( \beta \cos{(X - \alpha)} + (1 - \beta) \cos{(X + \alpha)} \right) \eta_1 = 0, \\
			m\ddot{\eta}_2 + \dot{\eta}_2 + \left( (1 - \beta) \cos{(X + \alpha)} - \beta \cos{(X - \alpha)} \right) \eta_1 = 0,
		\end{cases}
	\end{equation}
	
	
	\begin{equation} \label{split-linear-pert-sys-ksi-eta}
		\begin{cases}
			m\ddot{\xi}_n + \dot{\xi}_n + \left( (1 - \beta) \cos{(X + \alpha)} + \beta \cos{\alpha} \right) \xi_n = 0, \\
			m\ddot{\zeta}_n + \dot{\zeta}_n + \left( (1 - \beta) \cos{\alpha} + \beta \cos{(X - \alpha)} \right) \zeta_n = 0.
		\end{cases}
	\end{equation}
	
	Так как $X$ периодична, мы можем применить теорию Флоке
	и найти мультипликаторы, определяющие асимптотическое поведение решений системы.
	
	Проанализируем систему \ref{split-linear-pert-sys-n12} (переменные $\eta_1$, $\eta_2$).
	Из \ref{pre-pend} следует, что одним из
	решений первого уравнения является $\dot{X}$.
	Так как $X$ периодична, то один из мультипликаторов равен 1.
	Согласно формуле Лиувилля-Остроградского, второй мультипликатор первого уравнения равен $\exp{(-\frac{T_x}{m})}$.
	Кроме того, полная система имеет решение $\eta_1 = 0$, $\eta_2 = const$,
	откуда следует, что еще один мультипликатор равен единице.
	Вновь применяя формулу Лиувилля-Остроградского, но
	уже ко всей системе, находим четвертый мультипликатор,
	равный $\exp{(-\frac{T_x}{m})}$. 
	Итак, рассматриваемый режим всегда внутренне устойчив.

	
	Проанализируем систему \ref{split-linear-pert-sys-ksi-eta} (переменные $\xi$, $\zeta$).
	Переменная $\xi$ соответствует малому кластеру, переменная $\zeta$ соответствует большому кластеру. 
	Они не связанны между собой.
	Благодаря такому разделению переменных, мы можем проследить как меняется устойчивость
	двухкластерного режима с периодической расстройкой фаз для каждого кластера. В зависимости от $X$ устойчивость
	может пропасть или появится только у одного или сразу у двух кластеров, благодаря чему мы можем
	понять каким образом двухкластерный режим будет разрушаться в случае потери устойчивости.
	
	С помощью уравнений \ref{split-linear-pert-sys-ksi-eta} определим устойчивость стационарного
	двухкластерного вращательного режима, соответствующего состоянию $x_{p_1}$.
	Уравнения запишутся в виде:

	\begin{equation}
		x_{p_1} = 0, \ \alpha \in [-\pi/ 2, \pi/2]: 
		\begin{cases}
			m\ddot{\xi}_n + \dot{\xi}_n + \cos{\alpha}\xi_n = 0, \\
			m\ddot{\zeta}_n + \dot{\zeta}_n + \cos{\alpha}\zeta_n = 0.
		\end{cases}
	\end{equation}

	\begin{equation}
		\begin{split}
			x_{p_1} = 2\arcsin{\frac{(1 - 2\beta) \sin{\alpha}}{\sqrt{\cos{\alpha}^2 + (1 - 2\beta)^2\sin{\alpha}^2}}} - \pi , \alpha \in [\pi/2, 3\pi/2): \\ 
			:\begin{cases}
				m\ddot{\xi}_n + \dot{\xi}_n + A(\alpha, \beta) \xi_n = 0, \\
				m\ddot{\zeta}_n + \dot{\zeta}_n -A(\alpha, \beta) \zeta_n = 0.
			\end{cases}
		\end{split}
	\end{equation}
	Где $A$ - некоторый коэффициент. Заметим, что при $\alpha \in [-\pi/ 2, \pi/2)$, $x_{p_1}$ - устойчива в исходной системе,
	при $\alpha \in [\pi/2, 3\pi/2)$, $x_{p_1}$ - неустойчива в исходной системе.

	Получается, что в исходной системе двухкластерный вращательный режим с постоянной расстройкой фаз всегда является неустойчивым.
	Таким образом дальнейший анализ устойчивости мы будем проводить только для двухкластерных вращательных движений с периодической
	расстройкой фаз.

\end{chapter}
	\likechapter{Заключение}
В данной работе были рассмотрены двух
    кластерные вращательные движения 
    в системе Курамото с инерцией.
    Были определены типы возникающих двух кластерных
    режимов. 
    Исследовали вопрос существования и
    устойчивости двух кластерных вращательных
    режимов в зависимости от управляющий параметров.
    Аналитические результаты подтверждены прямым численным 
    моделированием.
    Также разработано web приложение, выполняющее
    прямое численное моделирование системы, доступное по ссылке

  \begin{thebibliography}{10}
    \def\selectlanguageifdefined#1{
    \expandafter\ifx\csname date#1\endcsname\relax
    \else\language\csname l@#1\endcsname\fi}
    \ifx\undefined\url\def\url#1{{\small #1}}\else\fi
    \ifx\undefined\BibUrl\def\BibUrl#1{\url{#1}}\else\fi
    \ifx\undefined\BibAnnote\long\def\BibAnnote#1{(#1)}\else\fi
    \ifx\undefined\BibEmph\def\BibEmph#1{\emph{#1}}\else\fi

    \bibitem{Hoppensteadt:Izhikevich}
    F. C. Hoppensteadt and E. M. Izhikevich, Weakly connected neural networks, vol. 126 (Springer Science Business Media, 2012).

    \bibitem{Tinsley:Nkomo}
    M. R. Tinsley, S. Nkomo, and K. Showalter, Nature Physics 8, 662 (2012).

    \bibitem{Ding:Belykh}
    J. Ding, I. Belykh, A. Marandi, and M.-A. Miri, Physical Review Applied 12, 054039 (2019).

    \bibitem{Dorfler:Chertkov}
    F. Dörfler, M. Chertkov, and F. Bullo, Proceedings of the National Academy of Sciences 110, 2005 (2013).
    
    \bibitem{Kuramoto}
    Y. Kuramoto, in International Symposium on Mathemat- ical Problems in Theoretical Physics (Springer, 1975), pp. 420–422.
    
    \bibitem{Strogatz} 
    S. H. Strogatz, Physica D: Nonlinear Phenomena 143, 1 (2000).
    


    
    \bibitem{Acebron:Bonilla}
    J. A. Acebrón, L. L. Bonilla, C. J. P. Vicente, F. Ri- tort, and R. Spigler, Reviews of Modern Physics 77, 137 (2005).
    
    \bibitem{Barreto:Hunt}
    E. Barreto, B. Hunt, E. Ott, and P. So, Physical Review E 77, 036107 (2008).
    
    \bibitem{Ott:Antonsen}
    E. Ott and T. M. Antonsen, Chaos: An Interdisciplinary Journal of Nonlinear Science 18, 037113 (2008).
    
    \bibitem{Hong:Chate}
    H. Hong, H. Chaté, H. Park, and L.-H. Tang, Physical Review Letters 99, 184101 (2007).

    \bibitem{Pikovsky:Rosenblum}
    A. Pikovsky and M. Rosenblum, Physical Review Letters 101, 264103 (2008).

    \bibitem{Maistrenko:Popovych}
    Y. Maistrenko, O. Popovych, O. Burylko, and P. Tass, Physical Review Letters 93, 084102 (2004).

    \bibitem{Dorfler:Bullo}
    F. Dörfler and F. Bullo, SIAM Journal on Applied Dy- namical Systems 10, 1070 (2011).

    \bibitem{Martens:Barreto}
    E. A. Martens, E. Barreto, S. Strogatz, E. Ott, P. So, and T. Antonsen, Physical Review E 79, 026204 (2009).

  \end{thebibliography}
\end{document}