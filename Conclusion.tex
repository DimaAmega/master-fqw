\likechapter{Заключение}
В первой главе был изучен вопрос о существовании двухкластерных
вращательных движений. В частности было получено уравнение,
описывающее границы областей существования интересующего вращательного 
движения (см. \ref{borders}). Было показано, что существует два типа
двухкластерных вращательных движений. Первый характеризуется постоянной
расстройкой фаз, второй характеризуется периодической расстройкой фаз.


Во второй главе был изучен вопрос об устойчивости двухкластерных
вращательных движений. Было показано, что двухкластерное вращательное
движение с постоянной расстройкой фаз является неустойчивым при любых значениях
управляющих параметров. Также благодаря замене переменных была получена система (см. \ref{split-linear-pert-sys-ksi-eta}),
независимо описывающая устойчивость каждого кластера. Благодаря чему были построены карты
устойчивости двух кластерного режима в области параметров $m$, $\alpha$ в зависимости от
параметра $\beta = K/N$. Было показано, что устойчивость
двухкластерного режима часто может теряться из-за потери устойчивости
у одного из кластеров, как малого, так и большого. При этом другой кластер
сохраняет свою устойчивость. Результаты были подтверждены в рамках прямого
численного моделирования.
