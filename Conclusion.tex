\likechapter{Заключение}
В рамках данной работы был проведен анализ двукластерных вращательных режимов в ансамбле глобально связанных фазовых осцилляторов с инерцией. 
В первой главе был изучен вопрос о существовании двухкластерных
вращательных движений. В частности было получено уравнение,
описывающее границы областей существования интересующего вращательного 
движения (см. \eqref{borders}). Было показано, что существует два типа
двухкластерных вращательных движений. Первый характеризуется постоянной
расстройкой фаз, второй характеризуется периодической расстройкой фаз.


Во второй главе был изучен вопрос устойчивости двухкластерных
вращательных движений. Было показано, что двухкластерное вращательное
движение с постоянной расстройкой фаз является неустойчивым при любых значениях
управляющих параметров. Также благодаря замене переменных была получена система (см. \eqref{split-linear-pert-sys-ksi-eta}),
независимо описывающая устойчивость каждого кластера. Благодаря чему были построены карты
устойчивости двухкластерного режима в области параметров $m$, $\alpha$ в зависимости от
параметра $\beta = K/N$. Было показано, что устойчивость
двухкластерного режима часто может теряться из--за потери устойчивости
у одного из кластеров, как малого, так и большого. При этом другой кластер
сохраняет свою устойчивость. Результаты были подтверждены в рамках прямого
численного моделирования.

В третьей главы был описан программный комплекс, позволяющий находить произвольные
вращательные движения в системах связанных фазовых элементов и определять их устойчивость. Представлено краткое описание главных алгоритмов.
Были описаны основные модули, а также
основные подходы, которые использовались при реализации.
Реализованный программный комплекс размещен на веб--сервисе для хостинга IT--проектов и их совместной разработки GitHub.
