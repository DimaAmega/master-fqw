\begin{chapter}{Программный комплекс}

	Программный комплекс, описываемый в этой главе, позволяет находить вращательные движения и
	определять их устойчивость в системах связанных элементов, которые описываются системой однородных
	дифференциальных уравнений. Основная идея была описана в предыдущей работе \cite{Khorkin}.

\section{Описание реализации}
Программный комплекс написан на языке программирования python с использованием таких
библиотек как scipy \cite{scipy}, numpy \cite{numpy} и numba \cite{numba}.
Были реализованы три главных численных метода: интегрирование систем методом Рунге-Кутты 4-го порядка, метод Ньютона для
многомерного случая а также расчет матрицы Монодромии. Самостоятельная реализация этих методов была необходима для достижения 
максимальной производительности. 
Особый интерес представляет библиотека numba, которая реализует процедуру 
компиляции python кода в машинный код во время исполнения (jit compilation).
В частности, компиляция во время исполнения выполняется для функций, описывающих динамику системы.
Благодаря этому нахождение вращательных движений и определение их устойчивости реализованно эффективно.
В программном комплексе присутствует набор тестов \cite{testing}, проверяющих корректность работы главных алгоритмов.
% \section{Интерфейс}

% \section{Интерфейс}

% \begin{verbatim}
% 	(function) find_limit_cycle: (RS, args, IC0, T0, phase_period: float = 2 * mt.pi,
% 									h: float = 0.001, eps: float = 0.001) -> tuple
% \end{verbatim}

% \begin{verbatim}
% 	(function) get_monogrommy_matrix: (rs_orig, rs_linear, q0_limit_cycle: ndarray,
% 	T, args_linear, args_orig, h_integrate: float = 0.001) -> ndarray
% \end{verbatim}


\section{Ссылки на ресурсы}
Исходный код: https://github.com/unn-dynamic-systems/rotary\_states/

Проект на pypi: https://pypi.org/project/rotary\_states/

Тестирование: https://github.com/unn-dynamic-systems/rotary\_states/actions/

\end{chapter}