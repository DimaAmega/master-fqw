\begin{chapter}{Модель Курамото. Двухкластерные вращательные режимы}
	\section{Модель Курамото с инерцией}

	\begin{equation} \label{main-sistem}
		m\ddot{\varphi}_i + \dot{\varphi}_i = \omega + 
		\frac{1}{N} \sum_{j = 1}^N \sin{(\varphi_j - 
		\varphi_i - \alpha)}, i = \overline{1, N}, 
	\end{equation}
	
	где $m$ - масса, $\omega$ - вращающий момент,
	$\alpha$ - фазовая задержка, $N$ - общее число элементов.

	Предполагается, что осцилляторы являются идентичными, с идентичной частотой $\omega$, массой $m$ и запаздыванием по
	фазе $\alpha \in [0, \pi)$ \cite{Sakaguchi} . В системе \ref{main-sistem} имеется синфазный режим, который локально стабилен для любого $\alpha \in [0, \pi/2)$
	и нестабилен для любого $\alpha \in [\pi/2, \pi)$ \cite{Acebron:Bonilla}. В результате связь определяется как
	притягивающая для $\alpha < \pi/2$ и отталкивающая для $\pi/2 \leq \alpha < \pi$.

	\section{Двух кластерные вращательные режимы. Существование и типы}
	
	Заметим, что двух кластерный режим описывается системой:
	
	\begin{equation} \label{two-cluster}
		\begin{cases}
			m\ddot{\psi}_1 + \dot{\psi}_1 = \omega + \frac{N-K}{N} \sin{(\psi_2 - \psi_1 - \alpha)} - \frac{K}{N}\sin{\alpha},\\
			m\ddot{\psi}_2 + \dot{\psi}_2 = \omega + \frac{K}{N} \sin{(\psi_1 - \psi_2 - \alpha)} - \frac{N - K}{N}\sin{\alpha},
		\end{cases}
	\end{equation}
	
	где $\psi_1$, $\psi_2$ - фазы первого и
	второго кластера соответственно, $K$ - количество элементов в малом кластере.
	Введем новую переменную $\beta = \frac{K}{N}$, характеризующую долю малого кластера относительно всех элементов. 
	Уравнение \ref{two-cluster} перепишется в виде:
	
	\begin{equation} \label{two-cluster-beta}
		\begin{cases}
			m\ddot{\psi}_1 + \dot{\psi}_1 = \omega + (1 - \beta) \sin{(\psi_2 - \psi_1 - \alpha)} - \beta\sin{\alpha}, \\
			m\ddot{\psi}_2 + \dot{\psi}_2 = \omega + \beta \sin{(\psi_1 - \psi_2 - \alpha)} - (1 - \beta)\sin{\alpha}.
		\end{cases}
	\end{equation}
	
	
	Введем замену $X = \psi_1 - \psi_2$, характеризующую расстройку фаз между двумя кластерами.
	Вычитая из первого уравнения второе, \ref{two-cluster-beta} запишется в виде:
	
	\begin{equation} \label{pre-pend}
		m\ddot{X} + \dot{X} = (1 - 2 \beta) \sin{\alpha} - \left[(1-\beta)\sin{(X + \alpha)} + \beta\sin{(X - \alpha)} \right].
	\end{equation}
	
	Заметим, что $X = 0$ соответствует реализации
	синфазного вращательного режима. Если $X \neq 0$,
	это соответствует реализации двух кластерного режима. 
	
	Еще раз произведем замену переменных в уравнении \ref{pre-pend}:
	\begin{align*}
	R^2 = (N - 2K)^2 \sin{\alpha}^2 + N^2 \cos{\alpha}^2, \\
	\rho = \sqrt{\frac{N}{m R}}, \ t = \hat{t} \frac{N}{\rho R}, \gamma = \frac{N - 2K}{R}\sin{\alpha}, \\
	\Phi = X + \delta, \ \delta = \arccos{(\frac{N}{R}\cos{\alpha})}. \\
	\end{align*}
	
	Мы получаем:
	
	\begin{equation} \label{pend}
		\frac{d^2 \Phi }{d\hat{t}^2} + \rho \frac{d\Phi}{d\hat{t}} + \sin{\Phi} = \gamma
	\end{equation}

	Полученное уравнение \ref{pend} хорошо изучено \cite{Andronov:Vitt}. 
	В зависимости от параметров $\rho$, $\gamma$, в системе \ref{pend}
	могут существовать два состояния равновесия:
	устойчивая точка $\Phi_e = \arcsin{\gamma}$ и седло
	$\Phi_s = \pi - \arcsin{\gamma}$, а также
	некоторое устойчивое периодическое вращательное движение.
	Данные соотношения определяются так называемой
	бифуркационной кривой Трикомми \cite{Andronov:Vitt}.

	Таким образом, в исходной системе могут существовать 2 типа
	двух кластерных вращательных режимов, первый характеризуются
	постоянной расстройкой фаз, второй характеризуется
	периодической расстройкой фаз.
	
	Область существования вращательного движения в уравнении \ref{pend}
	определяется соотношением:
	\begin{equation}
		\gamma \ge T(\rho),
	\end{equation}

	где $T(\rho) = \rho\frac{4}{\pi} - 0.305\rho^3$ - уравнение, аппроксимирующее кривую Трикомми \cite{Belykh:Brister}.
	Подставляя, получаем уравнения для границы области существования двухкластерного вращательного движения с
	периодической расстройкой фаз в области параметров $\alpha$, $m$:
	\begin{equation} \label{borders}
		m = \frac{1}{\sqrt{(1 - 2\beta)^2\sin{\alpha}^2 + \cos{\alpha}^2} \cdot T^{-1}(\frac{(1 - 2\beta)\sin{\alpha}}{\sqrt{(1 - 2\beta)^2\sin{\alpha}^2 + \cos{\alpha}^2}})}
	\end{equation}

	Заметим, что двухкластерный вращательный режим, соответствующий
	$\beta = 0.5$ не существует.


\end{chapter}