\begin{chapter}{Численные методы}
Приведем описание численных методов, используемых для поиска регулярных 
вращательных движений, существующих в базовой модели (\ref{npend}), и определения 
их линейной устойчивости. Для начала отметим, что излагаемая ниже процедура 
является фактически модификацией схемы нахождения замкнутых предельных циклов 
в нелинейных динамических системах. Введём обозначение \{$ \varphi_n(t)$\} здесь и далее -- 
вектор-строка, где $n=\overline{1,N}$.
Основная идея этого метода заключается в следующем.
Любой представитель \{$ \varphi_n(t)$\}, интересующего нас класса траекторий характеризуется 
прежде всего своим периодом $T$ (который, неизвестен и должен быть 
найден в конце вычислительной процедуры) и числом $ k $, определяющим то, сколько раз 
каждая составляющая из набора циклических координат \{$ \varphi_n(t)$\} изменится на 2$\pi$ за 
промежуток времени $T$. Основываясь на этом, рассмотрим отображение  $\{ \varphi_n(0), \dot{\varphi}_n(0) \} \to \{ \varphi_n(T), \dot{\varphi}_n(T) \} $
и сконструируем следующий вектор:
\begin{equation*}
	p(T,\{\varphi_{0n},\dot{\varphi}_{0n}\}) = \{ \varphi_n(T,\{ \varphi_{0n},\dot{\varphi}_{0n} \})  - \varphi_{0n} -2 \pi k, \dot{\varphi}_n(T,\{ \varphi_{0n},\dot{\varphi}_{0n} \})  - \dot{\varphi}_{0n} \},
\end{equation*}
где $\{ \varphi_n(t), \dot{\varphi}_n(t) \}$ -- решение системы (\ref{npend}) с начальными условиями $\{\varphi_{0n},\dot{\varphi}_{0n}\}$, т.е. \newline $ \{ \varphi_n(0), \dot{\varphi}_n(0) \}=\{\varphi_{0n},\dot{\varphi}_{0n}\} $

Используя определенную подобным путем многомерную функцию $p(T,\{\varphi_{0n},\dot{\varphi}_{0n}\})$ 
можно сформулировать условие, дающее 
возможность найти $\{\varphi_n(t)\} $ и $T$. 
Оно состоит в равенстве нулю всех компонент вектора $p(T,\{\varphi_{0n},\dot{\varphi}_{0n}\})$ 
В итоге, приходим к тому, что необходимо подобрать такие 
значения $T$ и $\{\varphi_n(t)\} $, которые позволят удовлетворить требованию

\begin{equation}\label{peq0}
	p(T,\{\varphi_{0n},\dot{\varphi}_{0n}\}) = 0.
\end{equation}
Другими словами можно сказать, что задача теперь 
состоит в поиске неподвижной точки отображения $\{ \varphi_n(0), \dot{\varphi}_n(0) \} \to \{ \varphi_n(T), \dot{\varphi}_n(T) \} $ с 
учетом цикличности $ \{\varphi_n(t)\} $. В силу инвариантности системы (\ref{npend}) 
относительно трансляции во времени, одну из величин $ \{ \varphi_{0n} \}$ можно 
занулить без потери общности, и сделать тем самым количество 
неизвестных и число соотношений в (\ref{peq0}) одинаковым. Для отыскания корней совокупности уравнений (3) целесообразно использовать метод Ньютона, т.к. он обладает высокой эффективностью.
Продолжая эти решения по параметру $\beta$ в интервале неустойчивости синфазного режима, можно проследить все семейство нетривиальных периодических движений и проанализировать их бифуркации.

Для изучения линейной устойчивости произвольных (2$\pi$-, 4$\pi$-, 8$\pi$- и т.д.) 
периодических вращательных движений (с учетом цикличности) динамической 
системы (\ref{npend}) введем малые возмущения 
$\delta \varphi_n(t)$:  $ \{ \varphi_n(t) = \phi_n(t) + \delta \varphi_n(t) \} $, где $ \phi_n(t) $ -- рассматриваемое периодическое движение.
В результате процедуры линеаризации получим следующие 
уравнения для возмущений $ \{ \delta \varphi_n(t) \} $:
\begin{equation*}
	\begin{gathered} 
		\delta \ddot{\varphi}_i + \lambda \delta \dot{\varphi}_i+ \cos{(\phi_s(t)})\delta \varphi_i =  K ( \delta \varphi_{i-1}- 2\delta \varphi_i + \delta \varphi_{i+1 } ), i = \overline{1,N}, \\
		\delta \varphi_N = \delta \varphi_{N+1}, \ \delta\varphi_1 = \delta \varphi_0.
	\end{gathered}
\end{equation*}

Дальнейший анализ может быть проведён в рамках теории Флоке. 
Устойчивость рассматриваемых движений определяется спектром 
собственных значений матрицы монодромии (оператора Флоке) $M(T)$, которая задается выражением:

\begin{equation*}
	\{\delta \varphi_n(T),\delta \dot{\varphi}_n(T)\}^T = M\{\delta \varphi_n(0),\delta \dot{\varphi}_n(0)\}^T. 
\end{equation*}

Собственные значения $ \mu_m $ (здесь и далее $m=\overline{1,2N}$ ) матрицы $M(T)$ 
являются мультипликаторами Флоке, которые связаны с показателями 
Флоке $q_m$ периодического решения $\{\phi_n(t)\}$ соотношениеями $m=exp(iq_m) $. 
Таким образом, для определения устойчивости каждого обсуждаемого 
движения достаточно вычислить $\mu_m$. Если $|\mu_m| \leq 1 $ для всех $m$, тогда 
вращательный режим линейно устойчив. Стоит отметить, что одно 
из собственных значений $\mu_m$ всегда должно быть строго равным 
единице, т.к. $\{ \dot{\varphi}_n(t) \}$ принадлежит семейству периодических движений 
(с учетом цикличности). Следовательно, появляется дополнительная 
возможность проверки того, что найденное с помощью описанной 
выше процедуры решение $\{ \phi_n(t) \}$ принадлежит обсуждаемому классу 
предельных вращений. Если хотя бы один из мультипликаторов 
Флоке $\mu_m$ расположен на комплексной плоскости за пределами 
единичной окружности, то вращательный режим является линейно 
неустойчивым.
\end{chapter}