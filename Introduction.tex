\likechapter{Введение}
Сети фазовых осцилляторов часто используются для моделирования
совместной динамики в различных биологических и
искусственных системах, начиная от нейронных сетей \cite{Hoppensteadt:Izhikevich} и
популяций химических осцилляторов \cite{Tinsley:Nkomo} до лазерных решеток \cite{Ding:Belykh}
и электрических сетей \cite{Dorfler:Chertkov}. Система
фазовых осцилляторов Курамото первого порядка \cite{Kuramoto,Strogatz} представляет собой широко адаптированную модель
сети фазовых осцилляторов, которая может демонстрировать сложную
пространственно-временную динамику при переходе от
некогерентности к полной синхронизации \cite{Acebron:Bonilla,Barreto:Hunt,Ott:Antonsen,Hong:Chate,Pikovsky:Rosenblum,Maistrenko:Popovych,Dorfler:Bullo,Martens:Barreto}.
Когда колебания в модели Курамото имеют неоднородные частоты, этот переход обычно
сопровождается частичной синхронизацией, которая возникает, когда система
распадается на кластеры когерентных и некогерентных осцилляторов
\cite{Acebron:Bonilla,Martens:Barreto,Laing}.

Модель Курамото второго порядка с инерцией обычно используется для описания сетей генераторов,
способных регулировать свои собственные частоты, как, например, в адаптивной частотной
модели синхронизации светлячков \cite{Ermentrout} и системах электросетей \cite{Tumash}.
Включение инерции приводит к двумерной внутренней динамике осциллятора,
тем самым делая кооперативную динамику модели Курамото второго порядка существенно более
сложной, чем ее классическая противоположная часть первого порядка.
Эта динамика, вызванная инерцией, включает сложные переходы от некогерентности
к полной синхронизации \cite{Tanaka:Review,Tanaka:Physica,Peron,Munyaev:Smirnov,Komarov:Gupta,Olmi:Navas,Barabash:Belykh},
бистабильность синхронных кластеров \cite{Belykh:Brister}, хаотическую межкластерную динамику \cite{Brister:Belykh},
химеры \cite{Olmi:Chaos, Maistrenko:Brezetsky, Medvedev:Mizuhara} и уединенные состояния \cite{Jaros:Maistrenko, Jaros:Brezetsky}.

В работе \cite{Belykh:Brister} аналитически изучили возникновение и сосуществование стабильных
кластеров в двухпопуляционной сети идентичных осцилляторов Курамото второго порядка.
Две популяции разных размеров $K$ и $M$ естественным образом разделились бы на два кластера,
где осцилляторы синхронизируются внутри кластера, создавая фазовый сдвиг между кластерами.
Анализ, выполненный в \cite{Belykh:Brister}, позволил получить необходимые и достаточные условия для стабильности
двухкластерной синхронизации с постоянной фазой, а
также предоставил условие устойчивости концепции для двухкластерной синхронизации с
вращающимся фазовым сдвигом. 

Целью данной работы является изучение двухкластерных вращательных режимов в системе Курамото
с инерцией и запаздывающей по фазе связью Курамото-Сакагути \cite{Sakaguchi}, для успешного достижения которой выделены следующие задачи: \\
1) Исследование существования и типов возникающих двухкластерных вращательных режимов в зависимости от управляющих параметров. \\
2) Исследование устойчивости двухкластерных вращательных режимов в зависимости от управляющих параметров. \\
3) Реализация программного комплекса, позволяющего эффективно находить интересующие вращательные движения, а также определять
их устойчивость в произвольных системах связанных элементов.
