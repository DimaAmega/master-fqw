\likechapter{Введение}
Сети фазовых осцилляторов часто используются для моделирования
совместной динамики в различных биологических и
искусственных системах, начиная от нейронных сетей \cite{Hoppensteadt:Izhikevich} и
популяций химических осцилляторов \cite{Tinsley:Nkomo} до лазерных решеток \cite{Ding:Belykh}
и электрических сетей \cite{Dorfler:Chertkov}. Система
фазовых осцилляторов Курамото первого порядка \cite{Kuramoto,Strogatz} представляет собой широко адаптированную модель
сети фазовых осцилляторов, которая может демонстрировать сложную
пространственно-временную динамику при переходе от
некогерентности к полной синхронизации \cite{Acebron:Bonilla,Barreto:Hunt,Ott:Antonsen,Hong:Chate,Pikovsky:Rosenblum,Maistrenko:Popovych,Dorfler:Bullo,Martens:Barreto}. Когда
колебания в модели Курамото имеют неоднородные частоты, этот переход обычно
сопровождается частичной синхронизацией, которая возникает, когда система
распадается на кластеры когерентных и некогерентных осцилляторов
[7, 14, 15]. В случае идентичных осцилляторов частичная
синхронизация может превратиться в химерные состояния,
которые представляют собой захватывающие паттерны, в которых даже
структурно идентичные осцилляторы могут распадаться на две, возможно,
асимметричные группы когерентных и некогерентных осцилляторов [16-19].
Химерные состояния были тщательно изучены в модели Курамото,
а также в других сетях колебательных систем [19-28], включая связанные химические генераторы [2], сети
метрономов [29], связанные маятники [30],
педали на мосту [31], оптические системы и лазеры [32]
и непрерывные среды [33, 34].

Целью данной работы является изучение двухкластерных вращательных режимов в системе Курамото
с инерцией и фазовой задержкой, для успешного достижения которой выделены следующие задачи: \\
1) Исследование существования и типов возникающих двухкластерных вращательных режимов в зависимости от управляющих параметров. \\
2) Исследование устойчивости двухкластерных вращательных режимов. \\
3) Реализация программного комплекса, позволяющего эффективно находить интересующие вращательные движения, а также определять
их устойчивость в произвольных системах связанных элементов.
