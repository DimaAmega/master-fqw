\likechapter{Введение}
Сети фазовых осцилляторов часто используются для моделирования
совместной динамики в различных биологических и
искусственных системах, начиная от нейронных сетей \cite{Hoppensteadt:Izhikevich} и
популяций химических осцилляторов \cite{Tinsley:Nkomo} до лазерных решеток \cite{Ding:Belykh}
и электрических сетей \cite{Dorfler:Chertkov}. Система
фазовых осцилляторов Курамото первого порядка \cite{Kuramoto,Strogatz} представляет собой широко адаптированную модель
сети фазовых осцилляторов, которая может демонстрировать сложную
пространственно-временную динамику при переходе от
некогерентности к полной синхронизации \cite{Acebron:Bonilla,Barreto:Hunt,Ott:Antonsen,Hong:Chate,Pikovsky:Rosenblum,Maistrenko:Popovych,Dorfler:Bullo,Martens:Barreto}.
Когда колебания в модели Курамото имеют неоднородные частоты, этот переход обычно
сопровождается частичной синхронизацией, которая возникает, когда система
распадается на кластеры когерентных и некогерентных осцилляторов
\cite{Acebron:Bonilla,Martens:Barreto,Laing}.

Модель Курамото второго порядка с инерцией обычно используется для описания сетей генераторов,
способных регулировать свои собственные частоты, как, например, в адаптивной частотной
модели синхронизации светлячков \cite{Ermentrout} и системах электросетей \cite{Tumash}.
Включение инерции приводит к двумерной внутренней динамике осцилляторов,
тем самым делая кооперативную динамику модели Курамото второго порядка существенно более
сложной, по сравнению с динамикой классической модели первого порядка.
Данная динамика, связанная с учетом инерции, включает сложные переходы от некогерентности
к полной синхронизации \cite{Tanaka:Review,Tanaka:Physica,Peron,Munyaev:Smirnov,Komarov:Gupta,Olmi:Navas,Barabash:Belykh},
бистабильность синхронных кластеров \cite{Belykh:Brister}, хаотическую межкластерную динамику \cite{Brister:Belykh},
химеры \cite{Olmi:Chaos, Maistrenko:Brezetsky, Medvedev:Mizuhara} и уединенные состояния \cite{Jaros:Maistrenko, Jaros:Brezetsky}.

В работе \cite{Belykh:Brister} аналитически было изучено возникновение и сосуществование устойчивых
кластеров в двухпопуляционной сети идентичных осцилляторов Курамото второго порядка.
Две популяции разных размеров $K$ и $M$ естественным образом разделяются на два кластера,
внутри которых осцилляторы синхронизируются, создавая фазовый сдвиг между кластерами.
Анализ, выполненный в \cite{Belykh:Brister}, позволил получить необходимые и достаточные условия устойчивости
двухкластерной синхронизации, характеризующейся постоянной разностью фаз между элементами кластеров, а
также предоставил условие устойчивости для двухкластерной синхронизации с
вращающейся разностью фаз.

В работе \cite{slava} были изучены так называемые одиночные состояния, они характеризуются тем, что элементы разделяются на два кластера,
в одном из которых находится 1 элемент. Также в данной работу были выявлены зоны существования и устойчивости одиночных состояний.
Целью данной работы является обобщение результатов работы \cite{slava}, заключающееся в изучении
произвольных двухкластерных вращательных режимов в системе Курамото
с инерцией и запаздывающей по фазе связью Курамото-Сакагути \cite{Sakaguchi}, для успешного достижения которой были поставлены следующие задачи: \\
1) Исследование существования и типов возникающих двухкластерных вращательных режимов в зависимости от управляющих параметров. \\
2) Исследование устойчивости двухкластерных вращательных режимов в зависимости от управляющих параметров. \\
3) Реализация программного комплекса, позволяющего эффективно находить интересующие вращательные движения, а также определять
их устойчивость в произвольных системах связанных элементов.
